\documentclass[twoside,twocolumn,9pt]{article}

\usepackage[super,sort&compress,comma]{natbib}
\usepackage[left=1.5cm, right=1.5cm, top=1.785cm,
bottom=2.0cm]{geometry}
\usepackage[english]{babel}
\usepackage[T1]{fontenc}
\usepackage{hyperref}
\usepackage{graphicx}
\graphicspath{{./figures}{../figures}}
\usepackage{xcolor}

%\usepackage{epstopdf}
\usepackage{epstopdf}
\author{Mateo Barria-Urenda, José Antonio Gárate}
\title{Title}
\date{}

\begin{document}

\maketitle

\abstract{}

\section{Introduction}

% Context

First synthesized in 2004 \cite{Novoselov_2004}, graphene has since
experienced an explosive growth in interest \cite{Randviir_2014}.  For
pristine graphene, it's interactions with other particles are mainly
due to Van der Waals forces and $\pi-\pi$ stacking. \cite{Zuo_2012} As
pristine graphene is chemically inert \cite{Eftekhari_2017} a
classical description of it is expected to suffice in simulations.
Multiple molecular dynamics studies on the interactions between
proteins and carbon based nanoparticles (CBNs) --such as graphene--
have been published \cite{Zheng_2003, Ge_2011, Zuo_2012, Chong_2015,
  Duan_2015, Shityakov_2015, Al_Qattan_2018,Puigpelat_2019,
  Gonz_lez_Durruthy_2020, Li_2020}

% Need

% Task

% Work


\section{Methods}

\subsection{Free Energy Methods}

The Helmholtz free energy of adsorption of an amino acid onto a
graphene layer ($\Delta A_{ads}$) can be obtained from the difference
in free energy of the Adsorbed and Free states:

\begin{equation}
\label{eq:Adsorption}
\Delta A^{ads} = A^{near} - A^{free}
\end{equation}

Where the near and free states are defined based on the
distance between the graphene layer and the $\alpha-$carbon of the
amino acid projected onto a vector normal to the graphene layer. For a
graphene layer prepared along the XY plane, this is equivalent to the
distance along the Z--axis. This distance will be defined as the
reaction coordinate $\xi$. The cut-off ranges for the near and far
states will be defined based on the PMFs (potentials of mean force)
along $\xi$ of the amino acids, which can be found in the
supplementary materials \textcolor{red}{TODO}.

The free energy of a state is related to the probability of the state
via:

\begin{equation}
\label{eq:A-from-p}
A^i = - k_B T \ln (p_i)
\end{equation}

Where $k_B$ is Boltzmann's constant and T is the
temperature. Replacing Eq.~(\ref{eq:A-from-p}) in
Eq.~(\ref{eq:Adsorption}) we get an expression to calculate the free
energy of adsorption from the probabilities of the near and free
states:

\begin{equation}
\label{eq:Aads-from-p}
\Delta A^{ads} = - k_B T \ln (\frac{p_{near}}{p_{free}})
\end{equation}

If the PMF along $\xi$ has been estimated for the discrete
states $\{\xi_1, \xi_2, ..., \xi_N \}$, the probability of a state
$\xi_{j}$ can be estimated with:

\begin{equation}
\label{eq:p-from-PMF}
p_{\xi_j} = \frac{e^{-\beta~\mathrm{PMF}(\xi_j)}}{\sum_{i=1}^N e^{-\beta~\mathrm{PMF}(\xi_i)}}
\end{equation}

where $\beta$ is the thermodynamic beta ($\frac{1}{k_{B}T}$). To
estimate the probability of a state with multiple possible $\xi$
values, the numerator is replaced with a sum over the individual
values of $\xi$. In this way, the probabilities of the adsorbed and
free states can be estimated with:

\begin{equation}
\label{eq:p-near}
p_{near} = \frac{\sum_{j \in near} e^{-\beta~\mathrm{PMF}(\xi_j)}}{\sum_{i=1}^N e^{-\beta~\mathrm{PMF}(\xi_i)}}
\end{equation}
\begin{equation}
\label{eq:p-free}
p_{free} = \frac{\sum_{j \in free} e^{-\beta~\mathrm{PMF}(\xi_j)}}{\sum_{i=1}^N e^{-\beta~\mathrm{PMF}(\xi_i)}}
\end{equation}

Replacing Eqs.~(\ref{eq:p-near})~and~(\ref{eq:p-free}) in
Eq.\~(\ref{eq:Aads-from-p}) we have an expression to calculate the
free energy of adsorption from a PMF of $\xi$, regardless of how the
PMF was obtained:

\begin{equation}
\label{eq:Aads-from-PMF}
\Delta A^{ads} = - k_B T \ln \left(\frac{\sum_{j \in near} e^{-\beta~\mathrm{PMF}(\xi_j)}}{\sum_{k \in free} e^{-\beta~\mathrm{PMF}(\xi_k)}}\right)
\end{equation}

or, equivalently:

\begin{equation}
\label{eq:Aads-from-PMF-p}
\Delta A^{ads} = - k_B T \ln \left(\frac{\sum_{j \in near} p_{\xi_j}}{\sum_{k \in free} p_{\xi_k}}\right)
\end{equation}
On the following sections, two different methods used to obtain a PMF along $\xi$ will be described.

\subsubsection{Umbrella Sampling}

\subsubsection{Well-Tempered Multiple Walker Metadynamics}

\subsection{Molecular Dynamics Simulation}

MD simulations were performed using the GROMOS11 \cite{Riniker_2011,
  Schmid_2012} and the NAMD 2.14 simulation packages
\cite{Phillips_2020}.  Details of the simulations will be presented
separately for each program.

\subsubsection{GROMOS}

The SHAKE algorithm \cite{Ryckaert_1977} was employed to constrain all
bonds to their reference values with a relative tolerance of
$10^{-4}$, allowing for a time-step of 2 fs using the leapfrog
algorithm \cite{Hockney_1977}.  For water, the SPC model was used
\cite{Berendsen_1981}.  Non-bonded interactions were computed using a
triple range cut-off. Interactions within the short-range cut-off of
0.8 nm were computed every time-step, from a pair-list that was
generated every 5 steps.  At these time points, interactions between
0.8 and 1.4 nm were also computed which were kept constant between
these updates.  A reaction-field contribution was added to
electrostatic interactions approximating for a homogeneous medium
outside the 1.4 nm long-range cut-off, employing the relative
permittivity of SPC water (61) \cite{Tironi_1995}. All interactions
were calculated using the GROMOS 54a8 potential energy function, with
all graphene atoms modeled as neutral sp$^2$ carbons \cite{Reif_2012}.
After a steepest-descent minimization to remove bad contacts, all
velocities were randomly assigned from a Maxwell-Boltzmann
distribution at 298 K.  All simulations were run coupled to
thermostats using the weak-coupling algorithm
\cite{Berendsen_1984}. The solute, graphene and solvent atoms were
independently coupled to different heat baths. Additionally, the
graphene layer was coupled to separate baths for regulation of its
center of mass motion and its rotational and internal degrees of
freedom. This totaled 4 heat baths.

\subsubsection{NAMD}





% \subsection{Quantum Mechanics / Molecular Mechanics}
% NAMD was used for QM/MM simulations due to it's hybrid simulation
% implementation \cite{Melo_2018}

\section{Results}

\subsection{Adsorption Free Energy}

\subsection{Adsorption Entropy}

\subsection{Adsorption Energy}

\subsection{Diffusion}

\section{Conclusions}


\section*{Acknowledgements}
This work was partially supported by grant no. ICM-Economia grant
no. P09-022-F Centro Interdisciplinario de Neurociencia de Valparaiso,
Universidad de Valparaiso; FONDECYT 1180987 (to J.A.G.), PAI grant
no. 77170045 (to J.A.G.) and a doctoral scholarship from
CONICYT--PFCHA/DOCTORADO BECAS NACIONAL/2020--21201020.  Access to the
supercomputing infrastructure of the National Laboratory for
High-Performance Computing was supported through grant no. ECM-02
(Powered@NLHPC).


\bibliography{refs.bib}
\bibliographystyle{plainnat}

\end{document}

%%%Local Variables:
%%% mode: latex
%%% TeX-master: t
%%% End:
